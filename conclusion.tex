\section{Conclusion}

\label{sec:conclusion}

Therefore going off what we discussed in this paper, it is apparent that hybrid encryption schemes allow for greater security than just using a symmetric key cryptosystem on its own and lends greater practical utility to public key cryptosystems and provides motivation for improving on them to make hybrid cryptosystems more secure, for otherwise enthusiasm for and practical adoption of public key encryption schemes might have been bottlenecked.

Through our discussions on a couple of implementations/ hybrid encryption schemes, it is clear how we can design key encapsulation and data encapsulation to suit particular needs or to gain particular security benefits, which leaves room for future innovation and improvement of existing schemes, not to mention that said discussion also threw some light on how it facilitates some very common daily life applications, viz. communication over the internet.

On a tangent, the idea and philosophy behind hybrid encryption is also being used to try and create encryption schemes that could provide classical security and at the same time be quantum safe, as a way of testing out such implementations prior to switching to entirely quantum-safe encryption schemes.
This is quite divergent from what we have discussed and we shall not stray into this, but interested readers may refer to \cite{csahybrid}.

Suffice all of this to say that we have just touched the tip of the iceberg on this subject, and this is an exciting field in itself to look into, with lots of prospects for innovation in the time to come.
