\section{Introduction and Motivation}
For quite some time leading up to the present, symmetric and asymmetric key cryptosystems have been used extensively for securing communications in networks and among various parties; and both have their own merits and drawbacks. We will not go deep down into the details of these types of cryptosystems, but instead briefly discuss those to set the stage for the object of our attention - hybrid encryption.

In the case of symmetric key encryption, the sender (Alice) and receiver (Bob) share the same key (and therefore must share the key somehow to enable secure connection and decryption of the ciphertext by the receiver) that only they are privy to, and the sender uses said key to encrypt the message and sends the ciphertext to the receiver, who decrypts it with the aid of his knowledge of the secret key. Any third party/adversary would not be able to obtain any information about the message being sent by looking at the ciphertext, if this is implemented properly.\cite{smirnoff_turner_2019}

Asymmetric key encryption, also known as public key encryption or two-key encryption\cite{kaliski_acry}, requires the use of a pair of keys, one public key and the other one being a private key. The public key can be made visible to anyone without compromising the security of the communication and is used to encrypt messages, but the private key is to be kept secret at all times as it is used to decrypt any messages encrypted using the public key.\cite{savvyasym}

Symmetric key encryption is often faster and more efficient than asymmetric key encryption, and therefore it is preferable to use the former for encrypting large volumes of data, though it has severe drawbacks, viz. the reuse of the secret key in a symmetric cryptosystem leaks some information about the key and renders its use precarious with repetitive use and degrades the security of the communication, as the adversary can use the information she (Eve) gathers by looking at ciphertexts generated by using the same key repetitively to learn some information about the key in question.\cite{smirnoff_turner_2019}

On the other hand, asymmetric key encryption is more secure in this regard as knowledge of the public key does not enable the adversary to figure out the private key at all, even if those two keys are mathematically related in the construction of the cryptosystem. However, this amazing property of asymmetric key cryptosystems comes at a cost - they take are less efficient than symmetric key cryptosystems computationally, and this tradeoff means that asymmetric key cryptosystems are rendered impractical for the purpose of encrypting large volumes of data, as mentioned earlier.

So the natural question is whether we can combine the efficiency of symmetric key encryption along with the strong security of asymmetric key encryption to produce a cryptosystem that gives us the best of both worlds? The answer is yes, and is found in the form of hybrid encryption.

\label{sec:intro}
