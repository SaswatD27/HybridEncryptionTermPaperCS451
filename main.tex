\documentclass[journal]{IEEEtran}
\usepackage[style=ieee]{biblatex}
\bibliography{references.bib}
\usepackage{amsmath}
\usepackage{authblk}
\usepackage{amssymb}
\usepackage{url}
\usepackage{graphicx}
\usepackage{float}
\usepackage{setspace}
\usepackage{amsthm}
\usepackage{algorithm}
\usepackage[none]{hyphenat}
\usepackage{algorithmic}
\usepackage{tgschola}
\usepackage{tcolorbox}

\newtheorem{theorem}{Theorem}
\newtheorem{corollary}{Corollary}[theorem]
\newtheorem{lemma}[theorem]{Lemma}

\theoremstyle{definition}
\newtheorem{definition}{Definition}

\theoremstyle{remark}
\newtheorem*{claim}{Claim}

\setlength{\parskip}{3mm}
\setlength{\parindent}{0mm}

% custom commands
\newcommand{\G}{\mathcal{G}}
\newcommand{\Zq}{\mathbb{Z}_q}
\renewcommand\qedsymbol{$\blacksquare$}

\begin{document}
\title{Hybrid Encryption: A Brief Introduction}

\author[1]{Ashish Panigrahi\thanks{All authors have contributed equally.}}\affil[1]{School of Physical Sciences\\NISER, Bhubaneswar}%
\author[2]{Sai Sriharsha Indukuri}\affil[2,3]{School of Mathematical Sciences\\NISER, Bhubaneswar}
\author[3]{Saswat Das}%\affil[1]{School of Mathematical Sciences\\NISER, Bhubaneswar}

\maketitle

% As a general rule, do not put math, special symbols or citations
% in the abstract or keywords.
%\onehalfspacing

\newcommand{\symm}{\text{symm}}

\section{Introduction and Motivation}

\label{sec:intro}

\subsection{The Need for Hybrid Encryption}

For quite some time leading up to the present, symmetric and asymmetric key cryptosystems have been used extensively for securing communications in networks and among various parties; and both have their own merits and drawbacks.
We will not go deep down into the details of these types of cryptosystems, but instead briefly discuss those to set the stage for the object of our attention - hybrid encryption.

In the case of symmetric key encryption, the sender (Alice) and receiver (Bob) share the same key (and therefore must share the key somehow to enable secure connection and decryption of the ciphertext by the receiver) that only they are privy to, and the sender uses said key to encrypt the message and sends the ciphertext to the receiver, who decrypts it with the aid of his knowledge of the secret key.
Any third party/adversary would not be able to obtain any information about the message being sent by looking at the ciphertext, if this is implemented properly \cite{smirnoff_turner_2019}.

Asymmetric key encryption, also known as public key encryption or two-key encryption \cite{kaliski_acry}, requires the use of a pair of keys, one public key and the other one being a private key.
The public key can be made visible to anyone without compromising the security of the communication and is used to encrypt messages, but the private key is to be kept secret at all times as it is used to decrypt any messages encrypted using the public key \cite{savvyasym}.

Symmetric key encryption is often faster and more efficient than asymmetric key encryption, and therefore it is preferable to use the former for encrypting large volumes of data, though it has severe drawbacks, viz. the reuse of the secret key in a symmetric cryptosystem leaks some information about the key and renders its use precarious with repetitive use and degrades the security of the communication, as the adversary can use the information she (Eve) gathers by looking at ciphertexts generated by using the same key repetitively to learn some information about the key in question \cite{smirnoff_turner_2019}.

On the other hand, asymmetric key encryption is more secure in this regard as knowledge of the public key does not enable the adversary to figure out the private key at all, even if those two keys are mathematically related in the construction of the cryptosystem.
However, this amazing property of asymmetric key cryptosystems comes at a cost - they take are less efficient than symmetric key cryptosystems computationally, i.e. they are computationally intense, and this trade-off means that asymmetric key cryptosystems are rendered impractical for the purpose of encrypting large volumes of data, as mentioned earlier.
This definitely hampered the usability of asymmetric key cryptosystems and their adoption.

So the natural question is whether we can combine the efficiency of symmetric key encryption along with the strong security of asymmetric key encryption to produce a cryptosystem that gives us the best of both worlds?
The answer is yes, and is found in the form of hybrid encryption.

\subsection{Basic Introduction to Hybrid Encryption}

So what we do with hybrid encryption is this - we use a symmetric key encryption scheme to encrypt the message itself, and then we use an asymmetric key encryption scheme to encrypt the key used in the symmetric key encryption part earlier.
The former component of hybrid encryption with symmetric key encryption is called the \textit{data encapsulation mechanism}, or DEM for short, and the latter with asymmetric encryption is called the \textit{key encapsulation mechanism}, or KEM for short \cite{kuro_he}.

This is implemented in the following way, given that the sender is named Alice and the receiver is named Bob.
(This has been described in great and more precise detail by Cramer and Shoup (2003) \cite{cryptoeprint:2001:108})

\begin{enumerate}
\item Alice obtains Bob's public key;
\item \textsc{DEM}: She then produces a fresh symmetric key $k_\text{symm}$ and then encrypts the message $m$ she wants to send using $k_\text{symm}$ to obtain a ciphertext $\chi$;
\item \textsc{KEM}: Note that only Alice possesses $k_\symm$ at this point in time, so she encrypts $k_\symm$ using Bob's public key (which is convenient, because symmetric keys in practice are usually and ideally of small and manageable sizes, and thus appropriate for public key encryption) to get $\psi$;
\item Alice sends $\bar\psi=(\psi,\chi)$ to Bob, and he decrypts it using his private key - first he verifies whether $\bar\psi$ properly encodes the tuple of ciphertexts $(\psi,\chi)$;

\begin{enumerate}
\item If not, then $\bar\psi$ is rejected and the process halts;
\item If yes, then $\psi$ is decrypted using Bob's private key; if this produces a reject, then the process is halted; if not, then he obtains the symmetric key $k_\symm$, using which he can decrypt $\chi$, the ciphertext pertaining to the message $m$, to get $m$ itself, if there's no issue that can possibly produce a reject.
\end{enumerate}

\end{enumerate}

What this essentially does is make the hybrid key encryption scheme essentially a public key (i.e. asymmetric key) encryption scheme and share in the security definitions of the same.\\

\section{Common Notions of Security}

\cite{shoup_cca}
The first thing we must do before testing the security of any encryption scheme, is to define what security is. The most widely accepted definition of security for an encryption scheme that involves Public Key Encryption is security against \emph{adaptive} chosen cipher-text attacks (introduced by Rackoff and Simon \cite{RS}).\\

\cite{springer_cca}
Chosen cipher-text attack (CCA) is a scenario in which the attacker has the ability to choose cipher-texts and to view their corresponding decryptions. Schemes that are secure against chosen cipher text attacks are called CCA secure.\\

An \emph{adaptive} chosen cipher-text attack is a chosen cipher-text attack scenario in which the attacker has the ability to make his or her choice of the inputs to the decryption function based on the previous chosen cipher-text queries.\\

It is clear from the two definitions that an adaptive chosen cipher text attack is stronger, and schemes that are secure against such attacks are, known to be CCA2-secure.\\

In the context of hybrid encryption,CCA security of the hybrid system as a whole boils down to the security of the Key Encapsulation Mechanism (KEM) (the scheme that encrypts the symmetric key with the receiver's public key) and the data encapsulation mechanism (DEM) (scheme that encrypts the message with the symmetric key). It has been proven that, if both the KEM and DEM are CCA secure, then the hybrid system as a whole is CCA secure. \cite{cryptoeprint:2001:108} 

(A few years later, it was also shown that CCA security of KEM is in fact not a necessary condition for the CCA security for the hybrid system. \cite{kuro_des})

A weaker notion of security, is security against chosen plain text attacks (aka CPA security). A chosen plaintext attack is when the attacker has the ability to choose messages and view their corresponding encryptions(the reverse of what happens in CCA).\\

The main advantage of systems that are CPA secure , when compared to CCA secure systems, is that CPA secure systems are very practical as computationally, they are very viable to implement. (This is the main reason for the popularity of schemes like ELGamal, which has been proven to be CPA secure but not CCA secure.)

\section{El Gamal}

\label{sec:elgamal} 

In this section, we proceed to describe a clever but efficient public key encryption scheme called \textit{El Gamal encryption} \cite{Elgamal_1985}. 
However, before diving into the details of the scheme, we provide a brief overview of necessary theorems in group theory.

\subsection{An overview of Group theory}

\begin{definition}
An Abelian group $ \mathcal{G} $ is a finite set of elements along with an operation $ * $ such that:

\begin{itemize}
    \item \textbf{Closure} For all $ a, b \in \mathcal{G} $ we have $ a * b \in \mathcal{G} $. We use $ a * b $ as $ ab $ henceforth for brevity.
    \item \textbf{Associativity} $ \forall a, b, c \in \mathcal{G} $, $ (ab)c = a(bc) $.
    \item \textbf{Commutativity} $ \forall a, b \in \mathcal{G} $, $ ab = ba $.
    \item \textbf{Existence of identity} $ \exists $ an element $ 1 \in \mathcal{G} $ such that $ 1 * a = a $ $ \forall a \in \mathcal{G} $. This element is the identity of $ \mathcal{G} $.
    \item \textbf{Inverse} $ \forall a \in \mathcal{G}$, $ \exists $ an element $ a^{-1} \in \mathcal{G} $ such that $ aa^{-1} = 1 $.
\end{itemize}
\end{definition}

\begin{theorem}
Let $\mathcal{G}$ be a finite Abelian group of order $ q $. Then $ a^q = 1 $ for all a $\in \mathcal{G}$.
\end{theorem}

\begin{proof}

Taken from \cite{koocmsc}.

Let us take $ a_1, \ldots a_q $ to be the elements of $\mathcal{G}$ and let $ a \in \mathcal{G} $ be some arbitrary element. We note that the sequence of elements $ aa_1, aa_2, \ldots , aa_q $ also contains exactly the elements of group $ \mathcal{G} $. Therefore:

\begin{equation*}
\begin{split}
a_1 \cdot a_2 \cdot \cdot \cdot a_q
&= (aa_1) \cdot (aa_2) \cdot \cdot \cdot (aa_q)\\
&= a^q (a_1 \cdot a_2 \cdot \cdot \cdot a_q)
\end{split}
\end{equation*}

On multiplying both sides by $ (a_1 \cdot \cdot \cdot a_q)^{-1} $, we get 

\begin{equation*}
a^q = 1
\end{equation*}
\end{proof}

\begin{corollary}
Let $ \mathcal{G} $ be a finite Abelian group of order $ q $, and let $ n $ be a positive integer.
Then $ g^n = g^{n \mod q} $.
\end{corollary}

\begin{proof}

Taken from \cite{koocmsc}.

Let us write $ n = n_q \mod q $ so that $ n $ can be further written as $ n = aq + n_q $ for some $ a \in \mathbb{Z} $. Then, we have

\begin{equation*}
g^n = g^{aq+n_q} = (g^a)^q g^{n_q} = g^{n_q}
\end{equation*}
\end{proof}

\begin{lemma}
If $ \mathcal{G} $ is an Abelian group with prime order $ q $, then 
\begin{enumerate}
    \item $ \mathcal{G} $ is cyclic; furthermore,
    \item every element of $ \mathcal{G} $ (except the identity) is a generator.
\end{enumerate}
\end{lemma}

\subsection{Discrete Logarithm Problem}

The problem can be described as follows: given a generator $ g $ of group $ \mathcal{G} $ and a random element $ h \in \mathcal{G} $, compute $ \log_g h $.

A common reference is made to the \textit{discrete logarithm assumption}, which says that for most groups, the discrete logarithm problem is ``hard".

\subsection{Description of El Gamal}

After the necessary overview, we go ahead and describe the El Gamal encryption scheme whose security is closely related to the discrete logarithm assumption:

\begin{algorithmic}
\STATE {\textbf{Key generation} \textsc{Gen}($ 1^k $):}
\STATE ($ \mathcal{G} $, $ q $, $ g $) $ \leftarrow \textsc{GroupGen} (1^k) $ 
\STATE Choose $ x \leftarrow \mathbb{Z}_q $; set $ y = g^x $ 
\STATE Output $ PK = ( \mathcal{G}, q, g, y) $ and $ SK = x $ 
\end{algorithmic}

\begin{algorithmic}
\STATE {\textbf{Encryption} $ \mathcal{E}_{pk} (m) $ (where $ m \in \mathcal{G} $)}
\STATE Pick $ r \leftarrow \mathbb{Z}_q $ 
\STATE Output $ \langle g^r, y^r m \rangle $ 
\end{algorithmic}

\begin{algorithmic}
\STATE {\textbf{Decryption} $ \mathcal{D}_{sk} (A, B) $}:
\STATE Compute $ m = \frac{B}{A^x} $ 
\end{algorithmic}

The correctness of the decryption process follows from,

\begin{equation*}
\frac{y^r m}{(g^r)^x} = \frac{y^r m}{(g^x)^r} = \frac{y^r m }{y^r} = m
\end{equation*}

\section{Secure Sockets Layer (SSL)}

One of the most common applications of hybrid encryption is the SSL protocol.

Secure Sockets Layer (SSL) is a security protocol used to establish an encrypted link between a server and a client (a website and a user's web browser for example). 
It allows private information like social security numbers, credit card numbers and email passwords to be transmitted securely from the client to the server.

Naturally, an encryption scheme to establish a link between a browser and a website's server must be as time efficient as possible. 
So asymmetric encryption doesn't seem to be a viable scheme to implement in such a setting, given its relatively high computation time. 
But at the same time, asymmetric encryption schemes are definitely more secure than, say, symmetric encryption schemes and this is highly desirable in an internet setting as users are often required to input sensitive information on various websites. 
This is precisely the reason hybrid encryption is used by SSL: we want a scheme that is both time efficient and highly secure.

\subsection{The Handshake}

When a browser attempts to access a website that is secured by SSL, the browser and the web server establish an SSL connection using a process called an “SSL Handshake” (the handshake, in fact, is invisible to the user and happens instantaneously) \cite{digicert}.

The handshake is implemented in the following way,
\begin{enumerate}
\item The browser connects to a web server secured with SSL and requests the sever to identify itself;
\item The server sends a copy of its SSL certificate, which includes the server's public key;
\item Browser checks the certificate root against a list of trusted \textit{certificate authorities} and that the certificate is unexpired, unrevoked, and that its common name is valid for the website that it is connecting to. If the browser trusts the website, it generates a random session key, encrypts it with the server's public key, and sends it back to the server;
\item The server decrypts the encrypted random session key with its private key and sends an acknowledgement message, encrypted with the session key, back to the client's browser;
\item The server and browser now communicate by encrypting messages with the session key.
\end{enumerate}  %SD-semicolons are used in lists like these professionally, though periods are also fine, but semicolons are preferred.

\label{sec:ssl}

\section{Conclusion}

\label{sec:conclusion}

Therefore going off what we discussed in this paper, it is apparent that hybrid encryption schemes allow for greater security than just using a symmetric key cryptosystem on its own and lends greater practical utility to public key cryptosystems and provides motivation for improving on them to make hybrid cryptosystems more secure, for otherwise enthusiasm for and practical adoption of public key encryption schemes might have been bottlenecked.

Through our discussions on a couple of implementations/ hybrid encryption schemes, it is clear how we can design key encapsulation and data encapsulation to suit particular needs or to gain particular security benefits, which leaves room for future innovation and improvement of existing schemes, not to mention that said discussion also threw some light on how it facilitates some very common daily life applications, viz. communication over the internet.

On a tangent, the idea and philosophy behind hybrid encryption is also being used to try and create encryption schemes that could provide classical security and at the same time be quantum safe, as a way of testing out such implementations prior to switching to entirely quantum-safe encryption schemes. This is quite divergent from what we have discussed and we shall not stray into this, but interested readers may refer to \cite{csahybrid}.

Suffice all of this to say that we have just touched the tip of the iceberg on this subject, and this is an exciting field in itself to look into with lots of prospects for innovation in the time to come.


\printbibliography

\end{document}
