\begin{abstract}
The dichotomy between symmetric key encryption and public key encryption is an oft discussed and studied topic in cryptography; 
and both come with their own sets of advantages and drawbacks, where symmetric key encryption is susceptible to being compromised by the repeated use of a symmetric key and public key encryption, for all its security benefits, is often too computationally intensive to encrypt plaintexts of a large volume. 
In our term paper, we introduce and discuss hybrid encryption, where these two forms of encryption are melded together to take advantage of the security benefits of public key encryption and the ease of use (computationally speaking) of symmetric key encryption.
We subsequently discuss two concrete examples of hybrid cryptosystems, viz. one based on ElGamal encryption (public key cryptosystem) and AES (a block cipher, and ergo symmetric key), and the popular SSL Handshake which is used to facilitate secure communication online.
We also delve briefly into some aspects of the security of hybrid cryptosystems.
\end{abstract}
